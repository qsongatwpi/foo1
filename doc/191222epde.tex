\documentclass[11pt]{amsart}
\usepackage{geometry}                % See geometry.pdf to learn the layout options. There are lots.
\geometry{letterpaper}                   % ... or a4paper or a5paper or ... 
%\geometry{landscape}                % Activate for for rotated page geometry
%\usepackage[parfill]{parskip}    % Activate to begin paragraphs with an empty line rather than an indent
\usepackage{graphicx}
\usepackage{amssymb}
\usepackage{epstopdf}
\DeclareGraphicsRule{.tif}{png}{.png}{`convert #1 `dirname #1`/`basename #1 .tif`.png}

\title{Discretization of Elliptic Linear PDE and Neural Network}
%\author{The Author}
%\date{}                                           % Activate to display a given date or no date

\begin{document}
\maketitle
%\section{}
%\subsection{}

\section{Problem setup}
\subsection{HJB}
We want to solve a d-dimensions linear PDE given below:
\begin{itemize}
 \item Domain 
 $$O = \{x\in \mathbb R^{d}: 0<x_{i}< 1, i =1,2, \ldots d\}.$$
 \item Equation on $O$: 
 $$(\frac 1 2 \Delta -  \lambda) v(x) + 
 \sum_{i=1}^db_i(x)  \frac{\partial v(x)}{\partial x_i}  
  + \ell(x) = 0.$$
 \item Dirichlet data on $\partial O$:
 $$v(x) = g(x).$$
\end{itemize}

\subsection{Examples} 

\subsubsection{Multidimensional PDE with quadratic function as its solution}
Consider a class of PDE with coefficients satisfying,
$$d - \lambda \|x - \frac 1 2 {\bf 1}\|_{2}^{2} 
+ b(x)\cdot (2x - {\bf 1})+ \ell(x) = 0,$$
where ${\bf 1}$ is an $\mathbb R^{d}$-vector with each element being $1$.
The exact solution is 
$$
v(x) = \|x - \frac 1 2 {\bf 1}\|_{2}^{2} = \sum_{i=1}^{d} (x_{i} - \frac 1 2)^{2}.
$$

\section{Discretization}

\subsection{FDM}
We introduce some notions of finite difference operators.
Commonly used first order finite difference operators 
are FFD, BFD, and CFD. 
Forward Finite Difference (FFD) is
$$\frac{\partial}{\partial x_{i}}v(x) \approx \delta_{he_{i}} v(x) 
:= \frac{v(x+he_{i}) - v(x)}{h}.$$
Backward Finite Difference (BFD) is
$$\frac{\partial}{\partial x_{i}}v(x) \approx \delta_{-he_{i}} v(x) 
:= \frac{v(x-he_{i}) - v(x)}{-h}.$$
Central Finite Difference (CFD) is
$$\frac{\partial}{\partial x_{i}}v(x) \approx 
\bar \delta_{h e_{i}} v(x)
:= \frac 1 2 (\delta_{-he_{i}} + \delta_{he_{i}}) v(x)
.$$
It can be verified that the CFD has the following explicit form:
$$\bar \delta_{h e_{i}} v(x) = \frac{v(x+he_{i}) - v(x-he_{i})}{2h}.$$
Second order finite difference operators are the followings:
$$
\frac{\partial^{2}}{\partial x_{i}^{2}} v(x)
\approx
\delta_{-he_{i}} \delta_{he_{i}} v(x)
= \frac{v(x+he_{i}) - 2 v(x) + v(x- he_{i})}{h^{2}}.
$$
Although the next operator will not be used below, we will write it for its completeness. If $i \neq j$, we use
$$
\begin{array}
 {ll}
\frac{\partial^{2}}{\partial x_{i} \partial x_{j}} v(x) &\approx
\frac 1 2 (\delta_{he_{i}} \delta_{-he_{j}} v(x) + 
\delta_{he_{j}} \delta_{-he_{i}} v(x))
\end{array}
$$


\subsection{CFD on PDE}
Approximations for PDE are 
$$
\frac{\partial v(x)}{\partial x_i} \leftarrow 
\delta_{\pm h e_{i}} v(x)
$$
and
$$
\frac{\partial^2 v(x)}{\partial x_i^2} \leftarrow
\delta_{-he_{i}} \delta_{he_{i}} v(x).$$
For simplicity, if we set 
$$
\gamma = \frac{d}{d+ h^{2} \lambda}, \
p^{h}(x \pm he_{i}|x) = \frac 1 {2d} (1 \pm h b_{i}(x)), \
\ell^{h}(x) = \frac{h^{2} \ell(x)}{d},
$$
then it yields DPP
$$
v (x) = \gamma 
\Big\{ \ell^{h}(x) + 
\sum_{i=1}^{d} 
p^{h}(x+he_{i}|x) v(x+he_{i})
+ p^{h}(x-he_{i}|x) v(x-he_{i})
\Big\}.
$$

\subsection{UFD on PDE}
Upwind finite difference(UFD) is the following:
$$
\frac{\partial v(x)}{\partial x_i} \leftarrow 
\delta_{ h e_{i}} v(x) \cdot I(b_{i}(x)\ge 0) +
\delta_{-he_{i}} v(x) \cdot I(b_{i}(x) <0)
$$
and
$$
\frac{\partial^2 v(x)}{\partial x_i^2} \leftarrow
\delta_{-he_{i}} \delta_{he_{i}} v(x).$$
Then, with
$$c = d+h\sum_{i} |b_{i}(x)|, \ 
\gamma = \frac{c}{c+h^{2}\lambda}, \ 
\ell^{h} = \frac{\ell(x) h^{2}}{c}, \ 
p^{h}(x \pm he_{i}|x) = \frac{1+ 2hb_{i}^{\pm}(x)}{2c},
$$
then it yields DPP
$$
v (x) = \gamma 
\Big\{ \ell^{h}(x) + 
\sum_{i=1}^{d} 
p^{h}(x+he_{i}|x) v(x+he_{i})
+ p^{h}(x-he_{i}|x) v(x-he_{i})
\Big\}.
$$



\end{document}  